\section{Ideen und Konzepte}

Der PEAX Command Line Client \texttt{px} soll zwei grundlegenden Design-Prinzipien folgen:

\begin{enumerate}
    \item der Unix-Philosophie, und
    \item dem \textit{Swiss Army Knive}-Ansatz, der bereits in \secref{sec:command-line-tools} beschrieben wurde.
\end{enumerate}

\subsection{Unix-Philosophie}

Die Unix-Philosophie wird oftmals mit den folgenden beiden Grundsätzen wiedergegeben \cite[12:51]{the-code-linux}:

\begin{enumerate}
    \item \textit{Everything is a file.}\footnote{Alles ist eine Datei.}
    \item \textit{When you build something, you write things that are for a single purpose, but that do that purpose well.}\footnote{Wenn man etwas erstellt, macht man es zu einem einzigen Zweck, den es gut erfüllen soll.}
\end{enumerate}

Für den ersten Grundsatz lassen sich schwer Quellen finden. Die Unix-Pioniere \textsc{Doug McIlroy} und \textsc{Brain Kernighan} beschreiben aber den bei der Entwicklung von Unix eingeschlagenen Weg mit einem hierarchischen Dateisystem als revolutionär und als Grund für den Erfolg von Unix und dessen Nachfolger \cite[Kapitel 4.1, S. 62-62, und Kapitel 9.1, S. 166]{unix-history-memoir}. Das experimentelle Betriebssystem Plan 9 verfolgte diesen Grundsatz noch konsequenter \cite[Kapitel 8.4, S. 161]{unix-history-memoir}.

Der zweite Grundsatz geht auf \textsc{Doug McIlroys} erste Stil-Maxime \cite[S. 3]{unixtimesharing} \textit{«Make each program do one thing well.»}\footnote{Jedes Programm soll eine Sache gut erfüllen.} zurück. Die zweite Maxime \textit{«Expect the output of every program to become the input to another, as yet unknown, program. Don't clutter output with extraneous information. [...]»} ist im Zusammenhang mit Kommandozeilenprogrammen ebenfalls zu beachten. \textsc{Doug McIlroys} \textit{Pipe}-Idee von 1964 und Ken Thompsons Implementierung davon 1972 haben dazu geführt, dass in Unix praktisch jedes Programm mit jedem anderen Programm zusammenarbeiten kann, sofern diese zweite Maxime eingehalten wird.\footnote{Ken Thompson soll den Pipe-Mechanismus nach der ersten Verwendung als «mind-blowing» bezeichnet haben \cite[Kapitel 4.4, S. 68-69]{unix-history-memoir}.} Auf die weiteren Maximen soll an dieser Stelle nicht eingegangen werden.\footnote{\textsc{McIlroys} dritte Maxime, man solle Software idealerweise innert Wochen gestalten und erstellen, um sie früh ausprobieren zu können, wird diese Arbeit bereits aufgrund ihres Rahmens und des vorangegangenen Prototypen gerecht.}

Jahre später ‒ und um etliche Erfahrung mit Unix reicher ‒ sammelte \textsc{Mike Gancarz} neun Grundsätze (\textit{tenets}) zur Unix-Philosophie, die hier zusammenfassend wiedergegeben werden \cite{unixphil}:

\begin{enumerate}
    \item \textit{Small is beautiful.} Grosse Programme entstehen dann, wenn das zu lösende Problem nicht vollends verstanden und eingegrenzt wird. Bei grossen Programmen wird eine signifikante Menge an Code geschrieben, die nicht das eigentliche Problem löst, sondern andere, nicht-essentielle Sachen macht. Kleine Programme hingegen sind einfach zu verstehen, einfach zu warten, benötigen weniger Systemressourcen ‒ und können einfacher mit anderen Programmen kombiniert werden.
    \item \textit{Make each program do one thing well.} Ein Programm soll nur zur Lösung eines Problems entwickelt werden, und nicht damit zusammenhängende oder gar ganz andere Probleme zu lösen versuchen. Dieses eine Problem sollte aber so gut wie möglich gelöst werden. Das Formatieren von Ein- und Ausgaben kann oft von anderen Programmen übernommen werden. Interaktive Programme können oft vermieden werden, indem die Aufbereitung der Eingabedaten mit einem anderen Programm gelöst wird.
    \item \textit{Build a prototype as soon as possible.} Eine Idee sollte zunächst anhand eines Prototypen getestet werden, bevor viel Aufwand in eine umfassende Spezifikation und anschliessende Implementierung fliesst, die nicht gebraucht wird. Das Problem ist in dieser frühen Phase oft noch nicht gänzlich verstanden, und ein entsprechendes Konzept würde bloss am Ziel vorbeischiessen. Ein früher Prototyp minimiert das Risiko für unnütze Aufwände und spart somit viel Zeit. Überhaupt entsteht gute Software über iterative Annäherungen an ein Ziel, nicht auf Basis eines einzigen Konzepts.
    \item \textit{Choose portability over efficiency.} Ein Programm, das auf vielen Plattformen läuft, ist besser als ein Programm, das die Vorzüge einer einzigen Plattform vollends ausnützt, jedoch nur auf dieser Plattform läuft. Performanceoptimierung für eine einzige Plattform führt oft dazu, dass ein Programm auf anderen Plattformen nur noch schlecht oder gar nicht mehr läuft. Es ist einfacher, eine plattformübergreifende Codebasis zu warten, als verschiedene plattformabhängige. Da die Hardware ständig schneller wird, lösen sich Performanceprobleme oftmals von selber ‒ sofern ein Programm nicht unnötig aufgeblasen wird.
    \item \textit{Store numerical data in flat ASCII files.} Daten müssen früher oder später auf ein anderes System übertragen werden. Bei Textdateien ist dies in der Regel kein Problem. Bei proprietären Binärformaten ist aber oftmals eine Konvertierung erforderlich, die Zeit und Geld kostet. Textdateien können problemlos manuell mit einem Texteditor und automatisiert mit anderen Programmen bearbeitet werden. \textit{Unix} bietet eine Vielzahl solcher Programme.\footnote{Heutzutage ist UTF-8 das am weitesten verbreitete Format für Textdateien. Dessen Erfolg geht nicht zuletzt darauf zurück, dass es mit dem Ziel entwickelt worden ist, eine Obermenge von ASCII zu sein, wodurch ASCII komplett vorwärtskompatibel zu UTF-8 ist. UTF-8 wurde von Rob Pike und Ken Thompson entwickelt.}
    \item \textit{Use software leverage to your advantage.} Programmierer sollen sich auf das eigentliche Problem konzentrieren, und nicht auf solche, die bereits von anderen Programmierern zufriedenstellend gelöst worden sind. Um dies zu ermöglichen, muss Programmcode zwischen Programmierern ausgetauscht werden, und nicht als streng gehütetes Geheimnis behandelt werden. Bestehende Software ‒ in der Form von Code oder kompilierter Software ‒ kann mit einer gewaltige Hebelwirkung für die eigene Arbeit eingesetzt werden.
    \item \textit{Use shell scripts to increase leverage and portability.} Mithilfe von Shell-Skripts kann in wenigen Zeilen Software genutzt werden, die hundertausenden Zeilen ‒ getestetem und von anderen Programmierern gewartetem ‒ C-Code entsprechen. Der Entwicklungszyklus von Shell-Skripts ist schneller als derjenige von C-Programmen, da der Kompilierungsschritt entfällt. Zudem sind Shell-Skripts portabler als Programme, die in C oder in einer anderen Hochsprache geschrieben sind.
    \item \textit{Avoid captive user interfaces.} Interaktive Programme haben ihre eigene Interaktionssprache, die sich von der Shell unterscheidet, und daher zunächst für jedes Werkzeug gelernt werden muss. Solche Programme gehen davon aus, dass der Benutzer ein Mensch ist, was die Verwendung via Skripts ‒ und somit die Hebelwirkung von Software ‒ verunmöglicht. Interaktive Programme tendieren dazu, Features aus der Umgebung, von der sie sich abkapseln, in eigener, meist schlechterer Implementierung anzubieten, was die Codebasis aufbläht und zu einem inkonsistenten Verhalten führt.
    \item \textit{Make every program a filter.} Die meisten Programme nehmen Eingabedaten entgegen, transformieren diese, und produzieren daraus Ausgabedaten. Sie produzieren keine originären Daten, sondern verarbeiten Daten, die zumeist von Menschen prodiziert worden sind. Programme, die als Filter konzipiert sind, werden diesem Umstand gerecht. Auf \textsc{Unix} werden Eingabedaten über die Standardeingabe (`stdin`) entgegengenommen und über die Standardausgabe (`stdout`) wieder ausgegeben. Mithilfe der Pipe können so beliebig lange Filterketten erstellt werden, wovon jedes Programm eine Transformation auf den Datenstrom vornimmt. Fehlermeldungen und andere Informationen, die sich von den Nutzdaten unterscheiden, sollen in die Standardfehlerausgabe (`stderr`) geschrieben werden, was eine gesonderte Verarbeitung solcher Meldungen ermöglicht.
\end{enumerate}

Für \texttt{px} werden diese Grundsätze folgendermassen angewendet:

\begin{enumerate}
    \item \textit{Small is beautiful.} \texttt{px} soll nicht jeden Anwendungsfall mit einem einfachen, benutzerfreundlichen Befehl abdecken, sondern mit folgender Doppelstrategie mit möglichst wenig Aufwand zu einem möglichst für alle Benutzergruppen befriedigenden Ergebnis kommen:
    \begin{itemize}
        \item Entwickler sollen eine möglichst weite hohe und feingranulare Abdeckung der API bekommen, indem \texttt{px} die HTTP-Methoden \texttt{GET}, \texttt{PUT}, \texttt{POST}, \texttt{PATCH} und \texttt{DELETE} anbietet, und quasi als PEAX-spezifisches \texttt{curl} mit transparentem Token-Handling fungiert.
        \item Andere Benutzergruppen sollen Befehle bekommen, die sie häufig benötigen und ihnen einen hohen Nutzen bringen, wie z.B. das rekursive Hochladen von Verzeichnissen mit Dokumenten.
    \end{itemize}
    \item \textit{Make each program do one thing well.} \texttt{px} soll sich nicht um die Inhalte der Payloads (Dokumente, Metadaten) kümmern und diese validieren oder auswerten, sondern nur sicherstellen, dass die Payloads an den richtigen Ort mit den richtigen Optionen übertragen werden. Beispielsweise soll das Durchsuchen und Auswerten von JSON-Payloads \textit{nicht} in \texttt{px} eingebaut werden, da es hierfür mit \texttt{jq} bereits ein sehr mächtiges Tool gibt.
    \item \textit{Build a prototype as soon as possible.} Ein minimaler Prototyp wurde bereits im Sommer entwickelt. Auf Basis dieses Prototypen wird nun \texttt{px} im Rahmen dieser Arbeit weiterentwickelt.
    \item \textit{Choose portability over efficiency.} Mit \textsc{Go} wurde eine Programmiersprache gewählt, die das Kompilieren für andere Betriebssysteme und Architekturen \textit{out of the box} unterstützt.\footnote{Mit \textsc{Ken Thompson} und \textsc{Rob Pike} sind zwei der drei Schöpfer dieser Programmiersprache Unix-Pioniere erster und zweiter Stunde, was man ihr an verschiedensten Stellen anmerkt.}
    \item \textit{Store numerical data in flat ASCII files.} Die unsicher verwahrten Tokens sollen in einer JSON-Datei namens \texttt{.px-tokens} im \texttt{HOME}-Verzeichnis des Benutzers abgelegt werden. JWT-Tokens sind numerische Daten im weitesten Sinn, sprich Base64-codiertes JSON. Zwar lässt sich JSON nicht wie eine «flache» ASCII-Datei komfortabel mit Tools wie \texttt{grep} und Konsorten bearbeiten, dafür bietet \textsc{Go} sehr komfortable \textit{Marshaling}-Mechanismen um Datenstrukturen aus und zu JSON zu serialisieren \cite[Kapitel 4.5]{gopl}. Wichtig ist, dass zur Speicherung der Tokens keine Binärdateien, sondern Textdateien \textit{im weitesten Sinn} (JSON) zum Einsatz kommen.
    \item \textit{Use software leverage to your advantage.} Die Hebelwirkung von \textit{bestehender} Software soll für \texttt{px} auf verschiedenen Stufen genutzt werden. Die sehr umfassende und hochwertige Standard Library von \textsc{Go}, gerade das sehr mächtige Package \texttt{net/http} unterstützt das Ansprechen einer RESTful HTTP-API. Eine Fremdkomponente von Zalando (\texttt{go-keyring}) bietet plattformübergreifende Unterstützung für den nativen Keystore des Betriebssystems. Das sehr mächtige \texttt{go}-Tool kommt u.a. zur statischen Quellcodeanalyse (\texttt{go vet}), zum Testen (\texttt{go test}), Verwalten von Abhängigkeiten (\texttt{go mod}) und kompilieren (\texttt{go build}) zum Einsatz. In der Entwicklung werden Hilfstools wie \texttt{golint} (Quellcodeanalyse), \texttt{goimports} (Formatierung von Code und Importieren von Packages) und \texttt{gopls} (automatische, Texteditor-unabhängige Code-Vervollständigung).
    \item \textit{Use shell scripts to increase leverage and portability.} Die Teststrategie (siehe \secref{sec:testing-q2}) setzt stark auf Shell-Skripts zum Erreichen einer hohen Testabdeckung mit aussagekräftigen Tests. Die Skripts könnten auch dann noch verwendet werden, sollte \texttt{px} dereinst mit einer anderen Programmiersprache neu implementiert werden (siehe auch \secref{sec:decision-programming-language}).
    \item \textit{Avoid captive user interfaces.} Grundsätzlich soll \texttt{px} interaktiv bedienbar und in Skripten verwendet werden können. So können Benutzername und Passwort per Kommandozeilenparameter mitgegeben werden, und werden nur interaktiv erfragt, falls ersteres unterlassen wird. Bei der Zwei-Faktor-Authentifizierung ist jedoch eine interaktive Eingabe vonnöten, da der SMS-Code zum Zeitpunkt der initialen Login-Anfrage noch nicht bekannt ist.\footnote{Mit einem One-Time Password würde dies theoretisch funktionieren, da die Codes jeweils vorweg für eine bestimmte Zeitspanne gültig sind. Der Nutzen hiervon ist jedoch äusserst gering, da der Benutzer ja doch am Terminal sitzen muss, um den Code \textit{vor} der Ausführung von \texttt{px} einzugeben.}
    \item \textit{Make every program a filter.} \texttt{px} soll keine unnötigen Ausgaben machen und nur Nutzdaten auf \texttt{stdout} ausgeben. Werden Vollzugs- und Erfolgsmeldungen benötigt, sollen diese per speziellem Parameter verlangt und nach \texttt{stderr} ausgegeben werden (siehe auch die Diskussion bei \secref{sec:feedback-sprint1}).
\end{enumerate}

\subsection{Swiss Army Knive}

Die «Swiss Army Knive»-Befehlsstruktur \cite[S. 290]{gopl}, wie sie im vorhergehenden Kapitel beschrieben (\secref{sec:command-line-tools}) und im Prototyp (\secref{sec:ausgangslage}) verwendet worden ist, soll beibehalten werden. Befehle in \texttt{px} sollen etwa folgendermassen aussehen:

\begin{lstlisting}[caption={Beispielhafter Befehl mit Command, Subcommand, Flags, Parametern}]
# Login-Vorgang: Tokens vom IDP holen
px login -env test -verbose -user john.doe@acme.org

# Upload-Vorgang: Ein Dokument hochladen
px upload -e test document.pdf

# Logout-Vorgang: lokale Tokens entfernen
px logout -e test
\end{lstlisting}

Die Struktur ist immer dieselbe: \texttt{[Befehl] [Unterbefehl] [Flags] [Parameter]}, wobei der \texttt{Befehl} im Rahmen dieser Arbeit immer \texttt{px} lautet. Die einzelnen Unterbefehle werden in den folgenden Abschnitten erläutert.

Die Flags treten in zweierlei Ausprägung auf: Einerseits als \textit{Switches}, wobei eine Option durch das Vorhandensein eines Flags aktiviert und durch das Fehlen deaktiviert wird. Das \texttt{-verbose}-Flag ist ein Beispiel dafür. Andererseits als \textit{Key-Value-Parameter}, wobei das Flag einen zusätzlichen Wert erfordert. Beispiele hierfür sind die Flags \texttt{-env} und \texttt{-user} im Beispiel oben.

Zu den Flags soll es jeweils eine lange Version (\texttt{-verbose}, \texttt{-user}, \texttt{-env}) und eine zugehörige kurze Version (bestehend aus dem ersten Buchstaben der langen Version) geben (\texttt{-v}, \texttt{-u}, \texttt{-e}). Treten dabei Kollisionen auf (mehrere Flags, die mit dem gleichen Buchstaben beginnen), muss improvisiert werden.

Parameter werden nicht von allen Befehlen erwartet/unterstützt. Beim \texttt{upload}-Unterbefehl ist etwa ein Dokument (als relativer Dateipfad) erforderlich. Bei anderen Befehlen sind es Ressourcenpfade.

\subsubsection{Befehle für das Token-Handling}

Die grundlegende Voraussetzung zur Interaktion mit der PEAX API und die wichtigste Erleichterung durch \texttt{px} ist das Token-Handling.

\subsubsection{Generische Befehle für Entwickler}

TODO: generische Befehle für Entwickler (GET, POST, PUT, PATCH, DELETE)

\subsubsection{Spezifische Befehle für Prototypen}

TODO: bequeme Befehle für Prototypen (archive)

\subsection{Token Handling}
\label{sec:konzept-token-store}

\subsubsection{Token Store}

TODO: token storage (sicher und unsicher)

\subsubsection{Token Refresh}

TODO: automatisches Retry
