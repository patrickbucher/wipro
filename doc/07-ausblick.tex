\section{Ausblick}

Nach drei Monaten Recherche, Konzeption, Implementierung, Testen und Dokumentation liegt \texttt{px} in Version 0.4.2 vor.\footnote{Nach dem letzten Meilenstein wurde noch ein Fehler behoben, der eine automatische Aktualisierung von sicher verwahrten Tokens beeinträchtigte. Zudem werden die Tagfarben nun besser randomisiert.} Der Sprung auf die Versionsnummer 1.0.0 wurde nicht vollzogen, zumal es sich hierbei um ein Zwischenergebnis handelt, und es noch viele nicht umgesetzte Ideen im Backlog gibt, deren Umsetzung \texttt{px} zu einem wesentlich «runderen» Anwendungserlebnis machen könnten.

\subsection{Reflexion der Arbeit}

Nach einem etwas holprigen Start mit einigen Unklarheiten in der Projekt- und Meilensteinplanung konnte bald mit der Umsetzung angefangen werden. Der bestehende Prototyp bot einerseits ein Framework, das die Entwicklung leitete, andererseits aber auch einiges an Code, der noch zu überarbeiten war. So konnten im ersten Sprint nicht alle geplanten User Stories umgesetzt werden. Die Investitionen in das Refactorig zu Beginn des Projekts zahlten sich aber im weiteren Verlauf aus, sodass bei den drei weiteren Sprints jeweils alle gesteckten Ziele erreicht werden konnten.

Das vorgängige Aufsetzen der Dokumentation mit \textsc{LaTeX} erwies sich als lohnende Investition. Es ist sehr motivierend, wenn Erweiterungen am Text per Knopfdruck\footnote{Für die Dokumentation wurde ebenfalls ein \texttt{Makefile} verwendet.} ein druckreifes Dokument zum Ergebnis haben. Die Nebendokumente (Backlog, Arbeitsjournal usw.) wurden mithilfe von \textsc{Markdown} und \textsc{Pandoc} erzeugt. Die meisten Grafiken wurden mithilfe von \textsc{Graphviz}\footnote{\url{https://www.graphviz.org/}} und \textsc{PlantUML}\footnote{\url{https://plantuml.com/de/}} erstellt. Dadurch konnten mit Ausnahme der Meilensteingrafik alle Artefakte in textueller Representation in die Versionskontrolle aufgenommen werden.

Dies erlaubt eine automatisierte Auswertung des Arbeitsjournals, welche die Aufwände pro User Story und Arbeitsbereich (allgemeine Projektaufgaben, Recherche, Dokumentation, Umsetzung) mithilfe eines \textsc{AWK}-Skripts aufsummiert.\footnote{Bisher wurden 65 Stunden in die Dokumentation, 26.5 Stunden in allgemeine Projektaufgaben, 7.5 Stunden in die Recherche und 76 Stunden in die Umsetzung investiert; d.h. total 175 Stunden Aufwand (Stand: Mittwoch, 18.12.2019).} Die \textsc{Unix}-Philoso\-phie mit ihrem Fokus auf die Verarbeitung von Plaintext war somit nicht nur ein Leitfaden für die Softwareentwicklung, sondern für das ganze Projekt.

Mit der gewählten Programmiersprache \textsc{Go} konnten sehr viele positive Erfahrungen gesammelt werden. Ja: das manuelle Error-Handling ist teils anstrengend, mühsam und führt zu repetitivem Code ‒ der jedoch mit etwas Disziplin grundsolide und sehr gut lesbar ist. Der Entwickler wird dazu gezwungen, sich mit jeder Fehlermöglichkeit einzeln auseinanderzusetzen, wodurch es selten Überraschungen ‒ und solche nur mit aussagekräftigen Fehlermeldungen gibt. Das oft monierte «Fehlen» von parametrischem Polymorphismus (vulgo «Generics») in \textsc{Go} stellte nie ein Problem dar.\footnote{Für das Schreiben von Libraries dürfte das Fehlen von Generics durchaus ein Problem darstellen; weniger bei Client-Anwendungen.} Während der Arbeit wurde \textsc{Go} viel besser kennengelernt; auch spezielle Features wie Linker-Flags und Build-Annotations, sowie die ganze Toolchain. \textsc{Vim} mit dem Plugin \texttt{vim-go}\footnote{\url{https://github.com/fatih/vim-go}} erwies sich als schlanke und komfortable Entwicklungsumgebung.

Die Arbeit wurde grösstenteils in der Freizeit durchgeführt; meistens frühmorgens vor der Arbeit im Büro. Die skriptbasierte Teststrategie, die ein funktionierendes Backend voraussetzt, führte einmal dazu, dass die Arbeit am Wirtschaftsprojekt unterbrochen werden musste, um im Büro die nicht funktionierende Testumgebung zum Laufen zu bringen.\footnote{Lauffähige Testumgebungen haben auch für das Tagesgeschäft eine höhere Priorität als die Weiterentwicklung von \texttt{px}.} Trotz solcher Episoden kann die gewählte Teststrategie wieder auf ähnliche Probleme angewendet werden, wenn auch mit einem etwas grösseren Fokus auf Unit Tests.

Insgesamt war das Wirtschaftsprojekt sehr lehrreich, wenn auch aufgrund der hohen Arbeitsbelastung sehr anstrengend. Für die übrigen Module mussten in diesem Semester viele Abstriche gemacht werden. Die inkrementell wachsende Software bot die nötige Motivation; das inkrementell wachsende Arbeitsjournal mit der automatischen Zeitauswertung sorgte für den nötigen Druck. Das Ergebnis ist zufriedenstellend, und bei der Arbeit wurde viel gelernt.

\subsection{Ungelöste Probleme}

Die Software \texttt{px} ist kein fertiges Produkt, sondern ein Projekt, das mit der Entwicklung des PEAX-Portals wachsen soll. Dies gelingt jedoch nur, wenn \texttt{px} auch verwendet wird. Durch eine fehlende Adaption (\secref{sec:Risikoanalyse}) dürfte \texttt{px} in kurzer Zeit verwaisen, da niemand mehr Interesse an dessen Weiterentwicklung hat. \texttt{px} kann nur überleben, wenn es eine solide Benutzerbasis über verschiedene Bereiche (Entwicklung, Testing, Projekte) etablieren kann.

Der Code von \texttt{px} ist in einem ordentlichen Zustand, hat aber noch Verbesserungspotenzial. Viele Parameter- und Rückgabelisten sind etwas lang geworden. Es gibt Codeabschnitte, die zwar nicht dupliziert worden sind, einander jedoch sehr ähnlich sehen. Hier besteht Potenzial für Refactoring.

Die skriptbasierte Teststrategie war für den Rahmen des Wirtschaftsprojekts ein pragmatischer Weg, mit verhältnismässig wenig Testaufwand eine hohe Testabdeckung zu erreichen. Es gibt jedoch noch einige Codestellen, die ohne Refactoring durch Unit Tests abgedeckt werden könnten. Auch der Befehl \texttt{px} selber liesse sich mit \textsc{Go}-Boardmitteln testen, d.h. ohne Skripts \cite[S. 308-310]{gopl}.

Die Weiterentwicklung von \texttt{px} ist noch nicht geklärt. In der bisherigen Form (Arbeit an Code und Dokumentation frühmorgens durch einen einzigen Entwickler) kann sie nicht weitergeführt werden, zumal im nächsten Semester eine Bachelor-Arbeit ansteht.

Die API von PEAX wird früher oder später erweitert oder gar durch eine neue Version ersetzt, worauf die Versionsansgabe \texttt{/v1/} in verschiedenen Ressourcenpfaden hindeutet. Zwar ist \texttt{px} dahingehend vorwärtskompatibel, dass es auch Ressourcenpfade mit anderen Versionsnummern verarbeiten kann, ändert sich aber die Struktur der Requests, ist Entwicklungsaufwand nötig.

\subsection{Weitere Ideen}

Das Backlog weist noch einige Punkte auf, die im Rahmen des Wirtschaftsprojekts nicht mehr umgesetzt worden sind. So kann die lokale Entwicklungsumgebung unterstützt werden, wenn sich die Entwickler auf Portnummern und Protokolle einigen. Mithilfe von Variablen in Ressourcenpfaden (z.B. \texttt{peaxId}) wären viele Endpoints einfacher ansteurbar und würden das Austauschen von Beispielen vereinfachen.

Im Rahmen des Wirtschaftsprojekts wurde viel Dokumentation erzeugt. Im Alltag ist ein schlankes \texttt{README} jedoch oft wichtiger. Um dieses Dokument schreiben zu können, muss zunächst die Zielgruppe etwas genauer abgesteckt werden. Soll \texttt{px} nur intern verwendet oder zusammen mit der PEAX-API nach aussen getragen werden? Soll der \texttt{README}-Text auf Deutsch oder auf Englisch verfasst werden?

Ein Ansatz für die Weiterentwicklung und Verbreitung von \texttt{px} wäre es, die Software als OpenSource auf \textsc{GitHub} freizugeben. Hier wären jedoch einige Fragen zu klären. Wie kann eine Community aufgebaut und betreut werden? Wer erhält nach welchen Kriterien welche Rechte? Wie soll mit Bugs umgegangen werden? Welcher Lizenz soll der Code unterstehen? Lösungsansätze zeigt \textit{Pieter Hintjens} auf, der massgeblich am Aufbau der \textsc{ZeroMQ}-Community beteiligt war, in seinem Buch \textit{Social Architecture} \cite{social-architecture}.
