\section{Ausblick}

Nach drei Monaten des Konzipierens, Implementierens und Dokumentierens liegt \texttt{px} in Version 0.4.1 vor.\footnote{Nach dem letzten Meilenstein wurde noch ein Fehler behoben, der eine automatische Aktualisierung von sicher verwahrten Tokens beeinträchtigte.} Der Sprung auf die Versionsnummer 1.0.0 wurde nicht vollzogen, zumal es sich hierbei um ein Zwischenergebnis handelt, und noch viele nicht umgesetzte Ideen im Backlog gibt, deren Umsetzung \texttt{px} zu einem wesentlich «runderen» Anwendungserlebnis machen könnten.

\subsection{Reflexion der Arbeit}

Nach einem etwas holprigen Start mit einigen Unklarheiten in der Projekt- und Meilensteinplanung konnte bald mit der Umsetzung angefangen werden. Der bestehende Prototyp bot einerseits ein Framework, das die Entwicklung leitete, andererseits aber auch einiges an Code, der noch zu überarbeiten war. So konnten im ersten Sprint nicht alle geplanten User Stories umgesetzt werden. Die Investitionen in das Refactorig zu Beginn des Projekts zahlten sich aber im weiteren Verlauf aus, sodass bei den drei weiteren Sprints jeweils alle gesteckten Ziele erreicht werden konnten.

Das vorgängige Aufsetzen der Dokumentation mit \LaTeX erwies sich als lohnende Investition. Es ist sehr motivierend, wenn Erweiterungen am Text per Knopfdruck\footnote{Für die Dokumentation wurde ebenfalls ein \texttt{Makefile} verwendet.} ein druckreifes Dokument zum Ergebnis haben. Die Nebendokumente (Backlog, Arbeitsjournal usw.) wurden mithilfe von \textsc{Markdown} und \textsc{Pandoc} erzeugt. Die meisten Grafiken wurden mithilfe von \textsc{Graphviz}\footnote{\url{https://www.graphviz.org/}} und \textsc{PlantUML}\footnote{\url{https://plantuml.com/de/}} erstellt. Dadurch konnten mit Ausnahme der Meilensteingrafik alle Artefakte in textueller Representation in die Versionskontrolle aufgenommen werden.

Dies erlaubt etwa eine automatisierte Auswertung des Arbeitsjournals, welche die Aufwände pro User Story und Arbeitsbereich (allgemeine Projektaufgaben, Recherche, Dokumentation, Umsetzung) mithilfe eines \textsc{AWK}-Skripts aufsummiert.\footnote{Bisher wurden 56.5 Stunden in die Dokumentation, 26 Stunden in allgemeine Projektaufgaben, 7.5 Stunden in die Recherche und 73.5 Stunden in die Umsetzung investiert; d.h. total 163.5 Stunden Aufwand (Stand: Montag, 16.12.2019).} Die \textsc{Unix}-Philoso\-phie war somit nicht nur ein Leitfaden für die Softwareentwicklung, sondern für das ganze Projekt.

Mit der gewählten Programmiersprache \textsc{Go} habe ich sehr viele positive Erfahrungen gemacht. Ja: das manuelle Error-Handling ist teils anstrengend, mühsam und führt zu repetitivem Code ‒ der jedoch mit etwas Disziplin grundsolide und sehr gut lesbar ist. Der Entwickler wird dazu gezwungen, sich mit jeder Fehlermöglichkeit einzeln auseinanderzusetzen, wodurch es selten Überraschungen ‒ und solche nur mit aussagekräftigen Fehlermeldungen gibt. Das oft monierte «Fehlen» von parametrischem Polymorphismus (vulgo «Generics») in \textsc{Go} stellte nie ein Problem dar.\footnote{Für das Schreiben von Libraries dürfte das Fehlen von Generics durchaus ein Problem darstellen; weniger bei Client-Anwendungen.} Während der Arbeit lernte ich \textsc{Go} viel besser kennen, auch spezielle Features wie Linker-Flags und Build-Annotations, sowie die ganze Toolchain. \textsc{Vim} mit dem Plugin \texttt{vim-go}\footnote{\url{https://github.com/fatih/vim-go}} erwies sich als schlanke und komfortable Entwicklungsumgebung.

Die Arbeit wurde grösstenteils in der Freizeit durchgeführt; meistens frühmorgens vor der Arbeit im Büro. Die skriptbasierte Teststrategie, die ein funktionierendes Backend voraussetzt, führte einmal dazu, dass ich die Arbeit am Wirtschaftsprojekt unterbrechen musste, um mich im Büro um die nicht funktionierende Testumgebung zu kümmern.\footnote{Lauffähige Testumgebungen haben auch für das Tagesgeschäft eine höhere Priorität als die Weiterentwicklung von \texttt{px}.} Trotz solcher Episoden würde ich die gewählte Teststrategie wieder auf ein ähnliches Problem anwenden, wenn auch mit einem etwas grösseren Fokus auf Unit Tests.

Insgesamt war das Wirtschaftsprojekt sehr lehrreich, wenn auch aufgrund der hohen Arbeitsbelastung sehr anstrengend. Für die übrigen Module musste ich in diesem Semester viele Abstriche machen. Die inkrementell wachsende Software bot die nötige Motivation; das inkrementell wachsende Arbeitsjournal mit der automatischen Zeitauswertung sorgte für den nötigen Druck. Mit dem Ergebnis bin ich zufrieden, und ich habe bei der Arbeit viel gelernt.

\subsection{Ungelöste Probleme}

- Adaption
- mögliches Refactoring: lange Parameterlisten, teils duplizierter Code
- evtl. bessere Testabdeckung möglich

\subsection{Weitere Ideen}

- siehe Backlog
- OpenSource
- mit der Entwicklung schritthalten
