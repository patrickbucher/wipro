\section{Evaluation und Validierung}

Das vorliegende Projekt entstammt dem PEAX-Ideation-Prozess, worin sich Entwicklungs- und Validierungsphase abwechseln. 

\subsection{Rückmeldungen von Entwicklern}

Nach jedem Sprint wird ein Inkrement an die Entwickler ausgeliefert (siehe \secref{sec:Vorgehen}). Die Rückmeldungen werden hier gesammelt und kommentiert ‒ und fliessen in den jeweils nachfolgenden Sprint ein.

\subsubsection{Sprint 1}
\label{sec:feedback-sprint1}

\begin{itemize}
    \item \textsc{Michael Buholzer} wünscht sich Erfolgs- und Vollzugsmeldungen nach dem Login oder dem Upload eines Dokuments.
    \begin{itemize}
        \item \texttt{stdout} sollte grundsätzlich «sauber» bleiben, d.h. frei von unnötigen Ausgaben, die ein nachgelagertes Programm wieder herausfiltern müsste. Eine wichtige Maxime von \textsc{Unix}-Programmen lautet: \textit{«Expect the output of every program to become the input to another, as yet unknown, program.»} \cite[S. 3]{unixtimesharing}. Siehe dazu auch \textit{Rule of Silence} \cite[S. 20]{unixart} und \textit{Silence is Golden} \cite[S. 111]{unixphil}.\footnote{Brian W. Kernighan berichtet von der Zeit, als die Pipe Einzug in \textsc{Unix} hielt, womit die Ausgabe eines Programms zur Eingabe eines anderen Programms gemacht werden konnte: \textit{«Ken [Thompson] and Dennis [Ritchie] upgraded every command on the system in a single night. […] Overall, the job was not hard—most programs required nothing more than eliminating extraneous messages that would have \textbf{cluttered a pipeline}, and sending error reports to stderr.»} \cite[S. 69]{unix-history-memoir}}
        \item \texttt{stderr} wird nicht nur als Ausgabekanal für Fehlermeldungen verwendet, sondern für Meldungen allgemein. Für Vollzugsmeldungen wäre \texttt{stderr} vorzuziehen.
        \item Da \texttt{stderr} in \texttt{px} bisher grundsätzlich für Fehlermeldungen verwendet wird, sollen Erfolgsmeldungen über ein zusätzliches Flag \texttt{-verbose}/\texttt{-v} aktiviert werden müssen.
        \item Bei anderen Anwendungsfällen signalisiert die Ausgabe des Payloads auf \texttt{stdout} den Erfolg der Operation. Beim Dokument-Upload besteht dieser beispielsweise aus der generierten UUID des hochgeladenen Dokuments.
    \end{itemize}
\item \textsc{Patrick Roos} sieht die Möglichkeit, \texttt{px} auch zur Handhabung der \textit{Vault Secrets}\footnote{\url{https://docs.ansible.com/ansible/latest/user_guide/vault.html}} (Verschlüsselung und Entschlüsselung von Benutzernamen, Passwörtern etc. zu verwenden.
    \begin{itemize}
        \item Im Arbeitsalltag von PEAX stellt die Handhabung von Vault Secrets tatsächlich eine teils mühsame und langwierige Aufgabe dar. Hier besteht durchaus Automatisierungsbedarf.
        \item \texttt{px} ist als «skriptbare» Anwendung für die PEAX API konzipiert und so potenziell für jeden PEAX-Anwender einsetzbar.
        \item Die Verwaltung und Verwendung von Vault Secrets ist hingegen eine Aufgabe im DevOps-Bereich und betrifft nur interne Entwickler bei PEAX.
        \item Eine der obersten Maximen von \textsc{Unix} lautet: \textit{«Make each program do one thing well. To do a new job, build afresh rather than complicate old programs by adding new ‹features.›} \cite[S. 3]{unixtimesharing} Die Verwaltung von Vault Secrets und das Ansprechen der PEAX API sind klar zwei verschiedene Sachen und somit nicht «one thing». Die genannte Idee muss also anderweitig weiterverfolgt werden.
    \end{itemize}
\end{itemize}

\subsubsection{Sprint 2}

\begin{itemize}
    \item \textsc{Stefano Pellegrini} fände einen Befehl \texttt{px version} sinnvoll, der die aktuelle Versionsnummer ausgibt. Damit könne man sicherstellen, dass man nicht etwa eine veraltete Version verwendet und (hinfällige) Rückmeldungen auf diese gibt.
        \begin{itemize}
            \item Tatsächlich stellen die meisten Kommandozeilentools eine solche Möglichkeit zur Verfügung, wie z.B. \textsc{Go} oder \textsc{Docker} mit den Befehlen \texttt{go version} und \texttt{docker version}. Andere Werkzeuge, wie etwa der Command Line Client von \textsc{Heroku} oder das Tool \textsc{Ripgrep} stellen ein entsprechendes Flag zur Verfügung (\texttt{heroku -v} bzw. \texttt{rg --version}).
            \item Ein Versionstag ist bereits über das Git-Repository verfügbar. Die jeweils aktuelle Version kann mittels \texttt{git describe --tags} abgerufen werden.
            \item Der Linker von \textsc{Go}\footnote{\url{https://golang.org/cmd/link/}} erlaubt es mit dem Parameter \texttt{-ldflags} uninitialisierte Strings im Programmcode mit einem Wert zu belegen.
            \item Heisst die Variable im \texttt{main}-Modul \texttt{Version}, kann die aktuelle Versionsnummer folgendermassen in die Binärdatei hineinkompiliert werden: \texttt{go build -ldflags="{}-X main.Version=\$(git describe --tags)"{} cmd/px.go}
        \end{itemize} 
    \item Weiter hat \textsc{Stefano Pellegrini} vorgeschlagen, dass \texttt{logout} sich auf die jeweils aktuelle Standardumgebung beziehen soll, damit man nicht immer eine Umgebung mit dem \texttt{-env}-Parameter angeben muss.
    \begin{itemize}
        \item Der Vorschlag ist sinnvoll und soll entsprechend umgesetzt werden.
    \end{itemize}
\item \textsc{Michael Buholzer} wünscht sich, dass die Ausgabe von JSON-Datenstrukturen automatisch formatiert wird (\textit{pretty print}).
    \begin{itemize}
        \item Grundsätzlich lässt sich die JSON-Ausgabe sehr einfach formatieren, indem sie mittels Pipe durch ein Programm wie \texttt{jq} gesendet wird\footnote{Siehe \url{https://stedolan.github.io/jq/}, das auch die Möglichkeit bietet, Teile mittels einer DSL aus der Datenstruktur zu extrahieren.}.
        \item \textsc{Go} bietet jedoch mit \texttt{json.Indent} eine sehr komfortable Funktion, womit ein beliebiger JSON-Payload\footnote{D.h. nicht nur ein JSON-Payload, dessen Struktur mittels einer \texttt{struct} und den entsprechenden Annotations beschrieben ist, was zu einem unverhältnismässigen Mehraufwand führen würde, zumal dann jeder mögliche Payload statisch beschrieben sein müsste.} einfach formatiert werden kann.
        \item Der Vorschlag soll umgesetzt werden. Weitere Features im Zusammenhang mit generischer JSON-Verarbeitung sollen jedoch der \textsc{Unix}-Philosophie entsprechend an \texttt{jq} oder ähnliche Programme mittels Pipe delegiert werden.
    \end{itemize}
\item Zudem schlägt \textsc{Michael Buholzer} im Zusammenhang mit dem \texttt{get}-Befehl vor, dass die PEAX ID automatisch anhand des eingeloggten Benutzers ergänzt werden soll, und man so nicht beispielsweise \texttt{profile/api/v3/profile/785.2120.8339.75} sondern bloss \texttt{profile/api/v3/profile} eingeben muss. Die PEAX ID sei für den Benutzer von \texttt{px} nirgends ersichtlich.
    \begin{itemize}
        \item Eine generische \texttt{GET}-Schnittstelle kann nur angeboten werden, wenn die Ressourcenpfade für den Client transparent sind. So ist es nicht möglich, die PEAX ID automatisch zu ergänzen, zumal sie nicht zwingend am Ende, sondern auch mitten im Ressourcenpfad auftreten kann.
        \item Eine bereits angedachte Lösung sind Variablen im Ressourcenpfad, die dann vom Client automatisch ergänzt werden, z.B. \texttt{profile/api/v3/profile/\{peaxid\}}. Hiermit kann eine generische Schnittstelle gewährleistet werden, und für den Benutzer wird die Handhabung einfacher.
        \item Evtl. wäre es sinnvoll, zu jeder Umgebung, die ein Login repräsentiert, ergänzende Token-Informationen ausgeben zu können, z.B. die PEAX ID.
    \end{itemize}
    \item \textsc{Stephan Korner} meldet eine Reihe spezifischer Use-Cases, die ihm beim Testen der Mobile App (\textsc{iOS}) nützlich sein würden:
        \begin{enumerate}
            \item den Dokumentstatus eines eingelieferten Dokuments (Agent API) setzen
            \item den Check-In-Prozess nach der Registrierung wiederholbar machen, d.h. zurücksetzen
            \item den Status von Rechnungen überschreiben
            \item die Handhabung von Organisationen ermöglichen
        \end{enumerate}
        \begin{itemize}
            \item Diese Funktionalitäten werden nach der User Story 13 (Einlieferung von Dokumenten per Agent API) für den ersten Punkt bzw. mit den User Stories 14-17 (generische Schnittstellen für \texttt{POST}, \texttt{PUT}, \texttt{PATCH}, \texttt{DELETE}) unterstützt.
        \end{itemize}
\end{itemize}

\subsubsection{Sprint 3}

\begin{itemize}
    \item \textsc{Patrick Roos} würde \texttt{px} gerne auch gegen seine lokale Entwicklungsumgebung verwenden können.
        \begin{itemize}
            \item Technisch ist dies möglich, die Umgebung \texttt{local} benötigt jedoch eine spezielle Konfiguration. So werden hier keine Domains verwendet, sondern bloss Hostnamen und Ports.
            \item Der lokale Proxy ist unter Port \texttt{8050}, der IDP unter Port \texttt{8080} verfügbar. Der Host heisst jeweils \texttt{localhost}.
            \item Das lokale Keycloak-Realm heisst \texttt{peax-id-local}.
            \item Die lokale \texttt{clientId} heisst \texttt{peax.local}.
        \end{itemize}
\end{itemize}
