\section*{Glossar}
\markright{Glossar}

Im Glossar werden einige PEAX-spezifische Begriffe erläutert. Auf allgemeine technische Begriffe, wie z.B. RESTful API oder OAuth 2.0, wird hier nicht eingegangen.

\begin{description}
    \item[Admin API] API zur Verwaltung der PEAX-Benutzer und zur Konfiguration benutzerübergreifender Einstellungen (z.B. Bankfeiertage, verfügbare Organisationen für Postabonnierungen), die über das sogenannte Backoffice von entsprechend befugten PEAX-internen Mitarbeitern verwendet wird.
    \item[Agent API] API für Zulieferer von PEAX (z.B. Scanning Center), womit den Benutzern Dokumente mit Metadaten in den Posteingang geliefert werden können.
    \item[Backoffice] Web-Anwendung für administrative Aufgaben, siehe \textit{Admin API}.
    \item[IDP] Identity Provider: Service, der die Benutzerkonti mit persönlichen Angaben, Identifikation, Authentifizierungsmechanismen (Passwörter, Tokens), Berechtigungen usw. verwaltet und diesen Authentifizierungsmechanismen zur Verfügung stellt.
    \item[PEAX ID] GTIN-13-Identifikation\footnote{\url{https://www.gin.info/}} eines PEAX-Benutzers, z.B. \texttt{123.4567.8901.23}. Wird oft als Teil des Ressourcenpfades von HTTP-Endpoints verwendet.
    \item[TOTP] Ein \textit{Time-based One-time Password} ist ein während eines kurzen Zeitintervalls (60 Sekunden) gültiges Passwort, das häufig zur Zwei-Faktor-Authentifizierung verwendet wird.
    \item[User API] API zur Verwaltung des persönlichen PEAX-Kontos (Posteingang, Uploads, abonnierte Organisationen usw.), die über das PEAX-Portal komfortabel bedient werden kann.
\end{description}
