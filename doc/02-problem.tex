\section{Problemstellung}

Die Problemstellung setzt sich einerseits aus dem gestellten Projektauftrag (siehe Anhang) und andererseits aus der damit implizierten Umgebung (Systeme, Technologien, Benutze, etc.) zusammen (siehe dazu auch \secref{sec:Systemkontext}).

\subsection{Analyse des Projektauftrags}

Der Projektauftrag beschreibt einen Command Line Client für eine RESTful-API. In diesem Zusammenhang gibt es Aspekte aus folgenden Bereichen zu analsysieren:

\begin{description}
    \item[Technologie] das Protokoll HTTP und der Authentifizierungsmechanismus OAuth 2.0
    \item[Server-Umgebung] die Umgebungen, die eine PEAX API anbieten
    \item[Client-Umgebung] die Benutzer, ihr Betriebssystem und ihre Kommandozeile
\end{description}

\subsubsection{Endpoints}

Eine RESTful-API besteht aus einer Reihe sogenannter \textit{Endpoints}, d.h Pfade zu Ressourcen, die abgefragt und/oder manipuliert werden können. Aus Platzgründen soll hier nicht auf einzelne Endpoints eingegangen werden. Beispielhaft erwähnenswert sind aber etwa der Token-Endpoint, bei welchem der Benutzer im Austausch seiner Credentials (Benutzername, Passwort und optionaler zweiter Faktor) ein \textit{Token Pair} holen kann; und der Document-Endpoint, auf welchen Dokumente hochgeladen werden können.

\subsubsection{HTTP-Methoden}

Ein Endpoint kann über eine oder mehrere HTTP-Methoden angesprochen werden \cite[Abschnitt 4.3]{RFC7231}. Im Kontext der PEAX API sind folgende Methoden relevant:

\begin{itemize}
	\item \texttt{GET}: Erfragt eine Repräsentation einer bestimmten Ressource; greift nur schreibend auf diese zu.
    \item \texttt{HEAD}: Analog zu \texttt{GET}, es wird jedoch nur der Header und nicht der Body der Ressource angefragt.
	\item \texttt{POST}: Übermittelt eine Ressource zur Speicherung oder Manipulation einer bestehenden Ressource.
	\item \texttt{PUT}: Ersetzt eine bestehende Ressource durch den mitgeschickten Payload.
	\item \texttt{DELETE}: Löscht eine Ressource.
    \item \texttt{OPTIONS}: Beschreibt die Kommunikationsoptionen für eine bestimmte Ressource, wird etwa für CORS Pre-Flight Requests eingesetzt \cite{mdn-cors}.
    \item \texttt{PATCH} Führt eine partielle Modifikation auf eine bestimmte Ressource aus. \cite{RFC5789} Die Modifikation wird in der Form \textit{JavaScript Object Notation (JSON) Patch} durchgeführt \cite{RFC6902}.
\end{itemize}

\subsubsection{HTTP Status-Codes}

Eine Antwort auf eine HTTP-Anfrage enthält jeweils einen Status-Code \cite[Abschnitt 6]{RFC7231}. Bei der PEAX API werden u.a. folgende Status-Codes häufig verwendet:

\begin{itemize}
	\item \texttt{200 OK}: Die Anfrage hat funktioniert.
	\item \texttt{201 Created}: Die Anfrage hat funktioniert, und dabei wurde eine neue Ressource erzeugt.
	\item \texttt{204 No Content}: Die Anfrage konnte ausgeführt werden, liefert aber keinen Inhalt zurück (etwa in einer Suche mit einem Begriff, zu dem keine Ressource gefunden werden kann).
	\item \texttt{204 Partial Content}: Der zurückgelieferte Payload repräsentiert nur einen Teil der gefundenen Informationen. Wird etwa beim Paging eingesetzt.
	\item \texttt{400 Bad Request}: Die Anfrage wurde fehlerhaft gestellt (ungültige oder fehlende Feldwerte).
	\item \texttt{401 Unauthorized}: Der Benutzer ist nicht autorisiert, d.h. nicht eingeloggt im weitesten Sinne.
	\item \texttt{403 Forbidden}: Der Benutzer ist zwar eingeloggt, hat aber keine Berechtigung mit der gewählten Methode auf die jeweilige Resource zuzugreifen.
	\item \texttt{404 Not Found}: Die Resource wurde nicht gefunden; deutet auf eine fehlerhafte URL hin.
	\item \texttt{405 Method Not Allowed}: Die Resource unterstützt die gewählte Methode nicht.
	\item \texttt{415 Unsupported Media Type}: Das Format des mitgelieferten Payloads wird nicht unterstützt. In der PEAX API sind dies etwa Dokumentformate, die beim Hochladen nicht erlaubt sind (z.B. \texttt{.exe}-Dateien).
	\item \texttt{500 Internal Server Error}: Obwohl die Anfrage korrekt formuliert und angenommen worden ist, kam es bei der Verarbeitung derselben zu einem serverseitigem Fehler.\footnote{In der PEAX API kommt es gelegentlich zu solchen Fehlern, die stattdessen mit dem Status \texttt{400 Bad Request} und einer aussagekräftigen Fehlermeldung beantwortet werden müssten. Wird z.B. bei der Einlieferung von Dokument-Metadaten eine syntaktisch fehlerhafte IBAN mitgegeben, tritt der Fehler erst bei der internen Verarbeitung, und nicht schon bei der Validierung der Anfrage auf. Hier besteht Handlungsbedarf aufseiten der Backend-Entwicklung.}
	\item \texttt{380 Unknown}: Dieser Status ist nicht Teil der HTTP-Spezifikation, wird aber nach einem Login-Versuch verwendet, wenn eine Zwei-Faktor-Authentifizierung (SMS, One Time Password) verlangt wird, und ist somit für die vorliegende Arbeit von hoher Relevanz.
\end{itemize}

\subsubsection{OAuth 2.0}

Im Hinblick auf das Wirtschaftsprojekt hat sich der Autor dieser Arbeit bereits im Vorsemester im Rahmen des Moduls \textit{Computer Science Hot Topics} (INFKOL) mit dem Thema OAuth 2.0 befasst \cite{infkol-oauth}. Detaillierte Informationen zu OAuth 2.0 können diesem Paper und den dort zitierten Primärquellen entnommen werden.

An dieser Stelle sollen nur die Grundlagen beschrieben werden, die dann im Umsetzungsteil (siehe \secref{sec:Realisierung}) bei Bedarf genauer eingegangen wird.

Bei einem Login-Vorgang mit OAuth 2.0 sendet der Benutzer seine Credentials (Benutzername, Passwort, optionaler zweiter Faktor wie SMS-Code) an den Token-Endpoint eines \textit{Identity Providers} (IDP). Stimmen diese Angaben mit den Informationen auf dem IDP überein, erhält der Benutzer ein \textit{Token Pair} bestehend aus \textit{Access Token} und \textit{Refresh Token}. Der Access Token dient zum Zugriff auf eine geschützte Ressource und ist in der Regel kurzlebig. (Bei PEAX läuft er nach fünf Minuten ab.) Mithilfe des Refresh Tokens kann sich der Benutzer einen neuen Access Token vom IDP holen. Der Refresh Token ist darum langlebiger (30 Minuten bei PEAX).

Es ist die Handhabung dieser Tokens (sehr lange, Base64-codierte Strings), die das Ansteuern der PEAX API mit Programmen wie \texttt{curl} und \textsc{Postman} so mühsam machen.\footnote{Eine beispielhafte Analyse ergab, dass ein Access Token 1604 Zeichen lang ist.}

\subsubsection{Umgebungen}

Bei PEAX gibt es verschiedene Umgebungen oder «Stufen», welche den ganzen PEAX-Stack (Datenbank, Backend, Frontend) für einen bestimmten Zweck zur Verfügung stellen:

\begin{description}
    \item[local] Die Entwickler können den PEAX-Stack zum Entwickeln und Testen lokal ausführen.
    \item[dev] Dies ist das Entwicklungssystem, worauf die Entwickler ihre Änderungen deployen, sobald diese vom jeweiligen Feature-Branch in den \texttt{develop}-Branch gemerged wurden. Diese Umgebung ist tendenziell sehr aktuell, aber dafür auch instabil.
    \item[test] Auf diese Stufe werden Änderungen übertragen, die auf Stufe \texttt{dev} erfolgreich getestet werden konnten. Diese Umgebung wird vom Product Owner zur Abnahme von User Stories verwendet, ist in der Regel eher stabil und repräsentiert nach jedem Sprint einen potenziell releasefähigen Stand.
    \item[stage] Diese Stufe dient für die Regressionstests. Hier wird nach jedem Sprint der letzte Stand von \texttt{test} übertragen. Bei einem Release wird der Stand von hier verwendet. Diese Umgebung ist stabil und jeweils maximal zwei Wochen alt.
    \item[prod] Von \texttt{stage} werden die Änderungen mehrmals pro Jahr (Ziel: einmal pro Monat) auf die Produktivumgebung übertragen. Dies ist die einzige Umgebung, auf der produktive Kundendaten abgelegt werden. Datenschutz und Sicherheit spielen auf dieser Umgebung eine besonders hohe Rolle.
    \item[devpatch] Dies ist die Entwicklungsumgebung für den Hotfix-Pfad. Nach einem Release wird der aktuelle Stand von \texttt{prod} auf diese Stufe deployed. Bis zum nächsten Release können hier dringende Fehlerkorrekturen vorgenommen werden.
    \item[testpatch] Dies ist die Testumgebung für den Hotfix-Pfad. Dringende Fehlerkorrekturen werden von \texttt{devpatch} auf diese Stufe übernommen und hier abgenommen. Die Änderungen werden von hier aus direkt auf \texttt{prod} deployed.
    \item[prototype] Hierbei handelt es sich um eine Umgebung, die sporadisch für Prototypen und Demos verwendet wird.
    \item[perf] Diese Umgebung wurde vor dem grossen v3-Release im Frühling für Performance-Tests verwendet und ist seither nur sporadisch in Betrieb.
\end{description}

\subsubsection{Arten von Parametern}

Ein HTTP-Request hat verschiedene Parameter: Dies sind einerseits Header-Parameter, wie z.B. \texttt{Content-Type}, womit der MIME-Type der Request-Bodys festgelegt wird, oder \texttt{Accept}, womit dem Server mitgeteilt wird, welcher MIME-Type der Response-Body haben soll \cite{RFC2616}.

Auch in der Authentifizierung und Autorisierung spielen Request-Header eine wichtige Rolle, zumal Access Tokens über per \texttt{Authorization}-Header an den Server übermittelt werden \cite[Kapitel 7.1]{RFC6794}.

Andererseits gibt es auch Query-Parameter, welche direkt an die URL angehängt werden. Letztere werden oft für die Navigation im Portal verwendet, zumal bei \texttt{GET}-Requests kein Request-Body übermittelt werden kann.

\subsubsection{Benutzer}
\label{sec:Benutzer}

Es gibt verschiedene Gruppen von Benutzern, die \texttt{px} gewinnbringend einsetzen können:

\begin{description}
    \item[Backend-Entwickler] Diese entwickeln, erweitern und korrigieren die RESTful-API, die das Backend von PEAX ausmachen. Von ihnen kann \texttt{px} einerseits für schnelle Tests und das Erstellen von Testdaten verwendet werden, andererseits kann \texttt{px} auch dabei hilfreich sein, die API (gerade Datenstrukturen) explorativ kennenzulernen.
    \item[Frontend-Entwickler] Ist die Spezifikation eines Endpoints unvollständig, unklar oder gar fehlerhaft, kann darauf kein funktionierendes Frontend aufsetzen. Hier kann \texttt{px} dabei hilfreich sein, das tatsächliche Verhalten des Backends zu überprüfen, und die Struktur der zurückgelieferten Payloads zu betrachten.
    \item[Tester] Die manuellen Regressionstests finden direkt auf dem Portal bzw. auf der App statt. Oftmals wäre es hilfreich, Testdaten grösseren Umfangs für einen neu registrierten Benutzer zu erstellen. Mithilfe von \texttt{px} können hierzu einfache Skripts zur Verfügung gestellt werden. (Die Skripts werden tendenziell eher von Entwicklern zur Verfügung gestellt, aber die Tester können diese nach Instruktion selbständig ausführen.
\end{description}

\subsubsection{Betriebssysteme}

Zu Beginn des Projekts (September 2019) waren auf den persönlichen Computern der Entwickler macOS und Windows im Einsatz. Mittlerweile (Stand Oktober 2019) wurden alle Windows-Rechner durch Geräte mit macOS ausgetauscht. Mit den Testern gibt es dennoch einige potenzielle Benutzer (siehe \secref{sec:Benutzer}), die \texttt{px} immer noch auf Windows einsetzen würden.

Auf zahlreichen virtuellen Maschinen der PEAX-Infrastruktur (etwa für Datenbanken) läuft Linux als Betriebssystem. Hier könnte \texttt{px} für verschiedene Service-Tasks (Monitoring, Alerting) eingesetzt werden.

Es sind somit die Betriebssysteme macOS, Windows und Linux für die Ausführung von \texttt{px} relevant. Was die Architektur betrifft, kommen derzeit nur Intel-Prozessoren mit 64-Bit-Architektur (\texttt{x86\_64}) zum Einsatz.

\subsubsection{Shells} 

Verschiedene Betriebssysteme haben verschiedene Shells. Im UNIX-Bereich gibt es auch zahlreiche Shells mit unterschiedlichen Merkmalen, die parallel zueinander installiert werden können.

Bash ist nicht nur die Standard-Shell vieler Linux-Distributionen, sondern kommt bei Entwicklern auch auf Windows als \textit{git Bash} zum Einsatz. Auf macOS gehört Bash ebenfalls zum Lieferumfang, wobei die mächtigere \texttt{zsh} seit macOS Catalina standardmässig zum Einsatz kommt \cite{macos-zsh}. Andere populäre UNIX-Shells wie Fish, Ksh und Tcsh haben zwar unterschiedliche Merkmale, jedoch den POSIX-Standard als kleinsten gemeinsamen Nenner \cite{posix-shell}.

Auf Windows spielen zudem die PowerShell sowie \texttt{cmd.exe} eine Rolle.

\subsection{Ausgangslage und Vorleistungen}

Das Projekt \texttt{px} wurde bereits am 11. Juni 2019 auf dem GitLab von PEAX erstellt\footnote{\url{https://gitlab.peax.ch/peax3/px}}. Als erstes wurde eine CI-Pipeline bestehend aus den Schritten `build` und `test` erstellt. Die Pipeline wurde mittels eines Dummy-Tests überprüft, der einmal erfolgreich durchlaufen und einmal scheitern sollte, um einen Positiv- und einen Negativtest durchführen zu können.

Es wurde eine Hallo-Welt-Programm im \texttt{cmd}-Unterverzeichnis \cite[p. 293]{gopl} erstellt, welches dazu diente, die Kompilierung für verschiedene Plattformen zu testen. Obwohl Go-Programme mittels `go build` kompiliert werden können und keine weitere Build-Konfiguration benötigen, wurde ein \texttt{Makefile} erstellt, das ausführbare Programme für verschiedene Plattformen im \texttt{build}-Unterverzeichnis erstellt, also z.B. \texttt{build/windows/px.exe} für Windows oder \texttt{build/linux/px} für Linux.

Das \texttt{Makefile} wurde später um ein \texttt{release}-Target erweitert, womit die kompilierten Artefakte jeweils in eine Zip-Datei verpackt werden, die den aktuellen Versionstag (z.B. \texttt{v0.0.3}\footnote{\url{https://semver.org/}}) im Dateinamen enthält.

Das Target \texttt{coverage} führt die Testfälle durch, misst die Testabdeckung und generiert eine HTML-Ausgabe des getesteten Codes. Rote Zeilen sind nicht durch einen Testfall abgedeckt, grüne Zeilen hingegen schon. \cite[Kapitel 11.3]{gopl}

Weiter sind bis am 31. Juli 2019 folgende Features implementiert worden:

\begin{description}
	\item[\texttt{px help}] zeigt eine einfache Hilfeseite auf der Kommandozeile an.
    \item[\texttt{px login}] führt einen Loginversuch mit den angegebenen Credentials durch. Benutzername und Passwort können entweder als Kommandozeilenparameter oder mittels interaktiver Eingabe (\texttt{stdin}) entgegengenommen werden. Im letzteren Fall wird das eingegebene Passwort nicht angezeigt, was mit einem externen SSH-Ter\-mi\-nal-Modul\footnote{\url{https://godoc.org/golang.org/x/crypto/ssh/terminal\#Terminal.ReadPassword}} erreicht wird. Bei einem erfolgreichen Login-Versuch werden \texttt{access\_token} und \texttt{refresh\_token} aus dem Response-Payload gelesen und im \texttt{\$HOME}-Verzeichnis des jeweiligen Betriebssystem-Benutzers in eine JSON-Datei namens \texttt{.px-tokens} abgespeichert.
	\item[\texttt{px logout}] löscht ein Token-Paar für eine bestimmte Umgebung. Pro Umgebung kann es zu jedem Zeitpunkt nur ein aktives «Login», d.h. Token-Paar geben. Es besteht auch die Möglichkeit, sämtliche Tokens auf einmal zu löschen. Hierbei wird \texttt{\$HOME/.px-tokens} nicht gelöscht, sondern nur das Property \texttt{tokens} geleert. (Die Datei enthält ein Initialisierungsdatum, das nicht verlorengehen soll.)
	\item[\texttt{px upload}] lädt eine Datei (z.B. PDF) auf das PEAX-Portal hoch. Diese Funktionalität wurde eingebaut, um die Funktionsweise von \texttt{px} vor dem Ideation-Gremium zu demonstrieren\footnote{Die hochgeladene Datei erschien Sekunden später im Web-Portal, was die Anwesenden von der Funktionsweise überzeugte}.
\end{description}

Für die Evaluierung des Prototypen werden zudem die Anzahl Aufrufe und das Datum des letzten Aufrufs von \texttt{px} in eine JSON-Datei \texttt{\$HOME/.px-usage} geschrieben.

Insgesamt wurden ca. 20 Arbeitsstunden in den Prototyp investiert. Ein grosser Teil des Codes kann für die Weiterentwicklung übernommen werden, muss jedoch umstrukturiert werden. So ist zuviel Logik im Hauptmodul \texttt{cmd/px.go}, die zwecks Wiederverwendbarkeit in das Library-Modul \texttt{px} überführt werden soll.

\subsection{Risikoanalyse}

TODO:

Projektrisiko: fehlende Adaption, Grösse der API

Sicherheitsrisiko: Payment-Schnittstelle, Token-Verwahrung

technisches Risiko: fehlende Usability, Implementierung scheitert
